%% LyX 2.3.4.2 created this file.  For more info, see http://www.lyx.org/.
%% Do not edit unless you really know what you are doing.
\documentclass[english]{article}
\usepackage[LGR,T1]{fontenc}
\usepackage[latin9]{inputenc}
\usepackage{color}
\usepackage{textcomp}

\makeatletter

%%%%%%%%%%%%%%%%%%%%%%%%%%%%%% LyX specific LaTeX commands.
\DeclareRobustCommand{\greektext}{%
  \fontencoding{LGR}\selectfont\def\encodingdefault{LGR}}
\DeclareRobustCommand{\textgreek}[1]{\leavevmode{\greektext #1}}
\ProvideTextCommand{\~}{LGR}[1]{\char126#1}


\makeatother

\usepackage{babel}
\begin{document}
\begin{flushright}
15 April 2021
\par\end{flushright}

Re: SUPE-D-20-01593 ``Making Legacy Fortran Code Type Safe through
Automated Program Transformation''

~

Dear Editor,

~

Below are the reviewers' comments with my responses.

~

Kind regards,

~

Wim Vanderbauwhede

~

~

---------

\subsection*{Reviewer \#1:}

The topic chosen is very interesting.

The related work could be more elaborate.

In the contribution the author is referring to ordinary Fortran complier,
which version of the compiler the author refers to and whether it
has been tested on various versions of Fortran compilers.

\textcolor{blue}{> I have added the compilers used and their versions
in �3 }

At the end of section 4.3 a summary of the entire section can be given
in a couple of lines.

\textcolor{blue}{> The final three lines of �4.3 are in fact the summary}

The complexity of the algorithms discussed for converting the given
program to type safe can be discussed.

\textcolor{blue}{> I have added the complexity of all algorithms}

The overhead incurred to convert a Fortran 77 to Fortran 90 and to
ensure that it is type safe needs to be discusses.

\textcolor{blue}{> We have added a discussion of this with the algorithms
for run-time checking. Essentially, there is no performance overhead
after conversion from Fortran 77 to Fortran 90; the run-time type
checks introduce the overhead of an if-condition to be checked, but
in practice this is negligible.}

A suggestion to have an optimized compiler which can do the above
mentioned job with backward compatibility.

\textcolor{blue}{> I am not quite sure what the reviewer means here
by ``backward compatibility''. As explained in the paper, we start
from Fortran 77 code but need to generate Fortran 90 code to guarantee
the type safety. It is not possible to have complete type-safety in
Fortran 77.}

I would like to accept this paper with minor revisions like analyzing
the complexity of the algorithm and the overhead required for conclusion.

\subsection*{Reviewer \#4}

I have read through this paper and I find it very interesting as the
majority of legacy code was in FORTRAN-77 and COBOL.

Although the paper is very interesting, I think it needs to be shortened
as it is too long. 

\textcolor{blue}{> I have moved most of the formal notation to Appendices,
this has made the paper both shorter and more readable.}

Also, the list of references does not seem to be complete nor adequate.

\textcolor{blue}{> I have added five more references}

I recommend accepting the paper, but after submitting a carefully
revised version of it.

\subsection*{Reviewer \#5}

Good morning dear Author,

first of all congratulations on the excellent work submitted to the
analysis.

In my analysis, I noted the following points below which made me decline
to approve the publication of your work:

1) The text made me understand that it is a guide for migration from
Fortran77 to Fortran90;

\textcolor{blue}{> I am sorry that the reviewer misunderstood: the
paper is not a guide for migration from Fortran77 to Fortran90. As
stated in the abstract, the paper presents an analysis of the type
safety of FORTRAN 77 and the novel program transformation and type
checking algorithms required to convert FORTRAN 77 subroutines and
functions into pure, side-effect free subroutines and functions in
Fortran 90. The migration to Fortran 90 is required to support the
features needed for type safety, but not the goal in itself. I have
clarified this in �3}.

2) The objectives of the work are not clear enough;

\textcolor{blue}{> I have added a sentence to the abstract to make
clear that the objective is to reduce errors by increasing type safety}

3) The experiments are not conclusive enough to prove the applicability
of the study.

\textcolor{blue}{> I have added a list of the programs on which the
approach was verified in �3}

\subsection*{Reviewer \#6: }

The subject of this paper is highly relevant and the compiler as outlined
in it will undoubtedly be very useful in making type safe Fortran
routines and functions available for accelerators that enhance HPC
systems.

It would have been nice when this objective somehow was reflected
in the title of the paper, but with the title being already quite
long I can understand that it has been omitted.

There are very few typos in the text and these can be easily hunted
down by the author. So, I will refrain from indicating them in this
review.

There are also some errors in the content that the author could easily
find by himself, but as they may be overlooked and of are importance
for the understanding, or ease of reading

we point them out, for sofar as catched by the reviewer.

1. Page 8, line 5 should be: x\_out,m = f x\_in,k

\textcolor{blue}{> OK}

2. Page 9, line 36: ...a character as (integer,1). => ...a character
as (character,1).

\textcolor{blue}{> (integer,1) is correct, we define a character as
a one-byte integer}

3. Page 10, line 15: Tuple = (\textgreek{t}\texttimes{} ...\texttimes \textgreek{t}\_i\texttimes{}
...\texttimes \textgreek{t}\_k) => Tuple =( \textgreek{t}\_1\texttimes{}
...\texttimes \textgreek{t}\_i\texttimes{} ... \texttimes \textgreek{t}\_k).

\textcolor{blue}{> OK}

4. Page 13, line 40: As explained below... => As explained above...

\textcolor{blue}{> OK}

5. Page 21, Algorithm 1: By placing the notation of the argument lists
starting at line 27 before the transformed

version f', f' could be written much more compact.

\textcolor{blue}{> I agree but I would prefer to keep the syntax Fortran-like,
in particular the order of the arguments would be different if I wrote
f'(a'',y)}

6. Page 21., line 46: I/O direction => INTENT

\textcolor{blue}{> OK}

7. Page 26, In the storage associations yl(2) is associated with xc(1),
xc(2) and not xl(1), xl(2)

yl(3) is associated with xc(3), xc(4) and not zl(1), zl(2)

zl(3) is associated with xc(7), xc(8) and not z1c, which does not
appear in the common of the caller.

zl(4) mght be associated with zx in the caller, but zx is not declared.

In all probability the common block in the caller should have read:
'common yc, xc, z1c, z2c'

\textcolor{blue}{> Thank you for pointing out this mistake, I have
corrected it}

By the way, it seems that here a convention is assumed in which variables
from the caller are suffixed with a 'c' while the

corresponding variables in the callee are suffixed with an 'l'. Later
on this convention seems to be maintained in the

subscripts of expressions, e.g., 'cseq\_c' and 'cseq\_l'. It would
be nice to mention such a convention explicitly and beforehand.

\textcolor{blue}{> I have added the convention before the start of
�4.6.1}

8. Page 26, line 37: 'decl' should be italic.

\textcolor{blue}{> OK}

9. Page 27, Algorithm 7, - formulas (2) and (3) the subscript 'e'
of idx\_l,e is not explained.

\textcolor{blue}{> I've added an explanation}

- the \textbackslash dot operator is not defined. Although one can
assume that 'adding to the set' is meant, it should be stated explicitly.

\textcolor{blue}{> I've added the definition}

- I can not find the removal of members from cseq\_l in the while
loop. So, in this form it should run indefinitely.

\textcolor{blue}{> I have added the removal of the members form cseq\_l
and cseq\_r}

10. In example 14, in function 't3 function f2(f)' the 'external f'
statement is missing.

\textcolor{blue}{> OK}

What makes this paper less appealing is the formal approach in section
4. For instance, the the definition of a pure function as given in
subsection 4.2 could be given in a few sentences and could be as unambiguous
as the formalism displayed in pages 7 and 8. 

\textcolor{blue}{> I have removed the notation from 4.2}

This superfluous use of set theoretical formalism is even more present
in subsections 4.4.1-{}-4.4.8. Virtually nothing in the content of
these subsections is used in the following sections, except in a very
restricted form in at the end of Algorithm 11. 

\textcolor{blue}{> I have moved most of the formal notation to Appendices,
so that the interested reader can refer to them but they do not break
the flow of the paper. }

I can think of no excuse to lessen the readability of the paper by
inserting these superfluous parts that may discourage readers to continue
to the more practical sections of your paper.

My recommendation is therefore to revise it in the sense that only
the really necessary notation is used and to introduce it properly
before it is applied.

\textcolor{blue}{> The reviewer's suggestion has been very helpful.
The readability of the paper has much improved by following them.}
\end{document}
